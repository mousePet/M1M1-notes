\section{Differentiation}
\emph{Geometrical definition.}
The derivative of a function $f$ at $x$ is the slope of the tangent to the graph $y= f(x)$ at $(x,f(x))$.
\begin{df}
\begin{align*}
f'(x) = \lim_{x \to 0} \frac{f(x+h) -f(x)}{h}
\end{align*}
$\frac{f(x+h) -f(x)}{h}$ denotes the slope of the secant through  $(x,f(x))$ and $(x+h,f(x+h))$.
\end{df}
\begin{rk}
$f'(x)$ is also a function with $\dom(f') \subseteq \dom(f)$.
\end{rk}
Using the limit definition to compute derivatives is called differentiation from first principles.
\begin{ex} \mbox \\
\begin{itemize}
\item  Polynomials
\begin{align*}
f(x) & = x^3 \\
\frac{f(x+h) -f(x)}{h} & = \frac{(x+h)^3-x^3}{h} \\
 & = \frac{x^3+3hx^2+3h^2x+h^3-x^3}{h} \\
& = 3x^2 + 3hx +h^2 \\
& \overset{x \to 0}{\to} 3x^2
\end{align*}
\item The cosine function
\begin{align*}
f(x) & = \cos x  \\
\frac{f(x+h) -f(x)}{h} & = \frac{\cos(x+h) - \cos x }{h} 
\end{align*}
Let us use the trigonometrical identity
\begin{align*}
\cos A - \cos B = -2 \sin \frac{A-B} 2 \sin \frac{A+B} 2
\end{align*}
(Derivation:
\begin{align*}
\cos(\alpha+\beta ) & = \cos \alpha \cos \beta - \sin \alpha \sin \beta \\
\cos(\alpha - \beta) & = \cos \alpha \cos \beta + \sin \alpha \sin \beta \\
\cos(\alpha+\beta ) - \cos(\alpha - \beta) & = -2\sin \alpha \sin \beta)
\end{align*}
With $A= x+b$ and $B=x$ this gives us
\begin{align*}
\frac{\cos(x+h) -\cos(x)}{h} & = \frac{-2 \sin \frac h 2 \sin \left( x + \frac h 2 \right)}{h} \\
& \overset{h \to 0}{\to} -\sin x
\end{align*}
\item
The function $f$ with 
\begin{align*}
f(x) = \left\{ 
\begin{array}{ll}
1, & x \in \mathbb Q \\
0, & x \in \R \backslash \Q
\end{array} \right.
\end{align*}
is not continuous for all $x \in \R$.
\end{itemize}
\end{ex}



\begin{tm}
If a function $f$ is differentiable at $a \in \dom f$ then $f$ is continuous at $a$. 
\end{tm}
This poses the question whether it is possible to find a function which is continuous but nowhere differentiable?
The Fourier series are
\begin{align*}
f(x) = \sum_{n=1}^\infty \frac{\sin(n\pi x)} n = \sin(\pi x) + \frac{\sin(2\pi x )}{2} + \frac{\sin(3 \pi x)}{3} + \dots 
\end{align*}
This led to the discovery of the Lacunary Fourier series
\begin{align*}
R(x) = \sum_{n=1}^\infty \frac{\sin(n^2\pi x)} n = \sin(\pi x) + \frac{\sin(4\pi x )}{2} + \frac{\sin(9 \pi x)}{3} + \dots 
\end{align*}
$R$ is not differentiable except for $x$ rational of the form $\frac p q$, $p,q$ odd.
\subsection{Basic Derivatives}
\begin{center}
\begin{tabularx}{.5\textwidth}{XX}
\toprule
$ f(x)$ & $f('x)$ \\
\toprule
 $x^n$ & $nx^{n-1}$ \\
\midrule
$\exp(x)$ & $\exp(x)$ \\
$\log x$ & $\frac 1 x$ \\
\midrule
$\sinh(x)$ & $\cosh (x) $\\
$\cosh (x)$ & $\sinh (x)$ \\
\midrule
$\sin(x)$ & $\cos(x) $\\
$\cos(x)$ & $-\sin(x)$ \\
$\tan(x)$ & $\sec^2(x)$ \\
\midrule
$\sin^{-1}(x)$ & $\frac 1 {\sqrt{1-x^2}} $  \\
$\tan^{-1}(x)$ & $\frac 1 {1+x^2}$ \\
\midrule 
$\sinh^{-1}(x)$ & $\frac 1 {\sqrt{1+x^2}} $ \\
$\tanh^{-1}(x)$ & $\frac 1 {1-x^2}$ \\
\bottomrule
\end{tabularx}
\end{center}
\subsection{Differentiation rules}
If $u,v$ and $f$ are derivable functions then the followings rules hold: 
\begin{itemize}
\item 
Addition rule
\begin{align*}
\frac d {dx} (u(x) + v(x) )= u'(x) + v'(x) 
\end{align*}
\item 
Multiplication rule
\begin{align*}
\frac d {dx} u(x) v(x) = u'(x) v(x)+u(x) v'(x) 
\end{align*}
\item 
Chain rule
\begin{align*}
\frac d {dx} f(u(x)) & = f'(u(x)) u'(x) \\
\frac{df}{dx} & = \frac{df}{du} \cdot \frac {du}{dx} 
\end{align*}
\end{itemize}
For the derivation of the product rule consider the following quotient:  
\begin{align*}
& \frac{u(x+h)v(x+h) -u(x)v(x)} h \\
= & \frac{u(x+h)v(x+h) - u(x)v(x+h) + u(x) v(x+h) - u(x)v(x)} h \\
= & v(x+h) \frac{u(x+h)-u(x)}{h} + u(x) \frac{v(x+h)-v(x)} h \\
\overset{h \to 0} \to & v(x)u'(x) +u(x)v'(x)
\end{align*}
The proof of chain rule will be done in spring term analysis.

\subsection{Implicit Differentiation}
\begin{rk}
Implicit differentiation applies the chain rule .
\end{rk}
\begin{ex}
Compute the slope of tangent to unit circle $x^2+y^2=1$.
\end{ex}
'Solve' to get $y=y(x)$ and use the differentiation rules.
\begin{align*}
y(x) & = \pm \sqrt{1-x^2} \\
y'(x)& = \pm \frac{-x}{\sqrt{1-x^2}} = \mp \frac x {\sqrt{1-x^2}}
\end{align*}
Implicit differentiation. Treat $y^2$ as a composite function -- differentiate with the chain rule.
\begin{align*}
\frac{d}{dx} y^2(x) = 2 y(x)y'(x)
\end{align*}
equation $x^2+y^2=1$. Differentiate with respect to $x$
\begin{align*}
2x + 2 y y' = 0 \quad \vee \quad y'=\frac{-x} y
\end{align*}


\begin{ex}
\begin{align*}
y^3-y &= x^2 \\
(3y^2 -1) y' & = 2x
\end{align*}
The slope of the tangent is 
\begin{align*}
y' = \frac{2x}{3y^2-1}
\end{align*}
For the point $(\sqrt 6, 2)$ we get the slope 
\begin{align*}
y' = \frac{2 \sqrt6}{11}
\end{align*}
\end{ex}
\subsection{Parametric Differentiation}
You can describe a curve in the $xy$ plane parametrically.
\begin{ex}
\begin{itemize}
\item 
Let us consider a curve defined by the hyperbolic functions.
\begin{align*}
x(t) & = \cosh(t) \\
y(t) & = \sinh (t), \qquad t \in \R
\end{align*}
slope of the tangent
\begin{align*}
\frac{dy}{dx} = \frac{\frac{dy}{dt}}{\frac{dx}{dt}} = \frac{\dot{y}}{\dot x}
\end{align*}
$\cdot $ denotes differentiation with respect to the parameter $t$.
\begin{align*}
\frac{dy}{dx} = \frac{\cosh t}{\sinh t} = \coth t
\end{align*}
\item 
The equation for a cycloid is
\begin{align*}
x(t) & = t - \sin t \\
y(t) & = 1- \cos t, \qquad t \in \R
\end{align*}
The point on the edge of a rolling wheel traces a cycloid.
\begin{align*}
\frac{dy}{dx} = \frac{\dot y}{\dot x} = \frac{\sin t }{1- \cos t}
\end{align*}
\end{itemize}
\end{ex}

\subsection{Higher Differentiation}
Suppose $f$ is differentiable then consider the limit
\begin{align*}
\lim_{h \to 0} \frac{f'(x+h) -f'(x)}{h}
\end{align*}
If this exists, $f$ is said to be twice differentiable. The limit is called second derivative, denoted 
\begin{align*}
f''(x) \quad \text{or} \quad \frac{d^2f(x)}{dx^2} \quad \text{or} \quad
y''(x) \quad \text{or} \quad \frac{d^2y(x)}{dx^2}
\end{align*}
This can be continued to define the $n^{th}$ derivative, denoted as
\begin{align*}
f^{(n)} (x) \quad \text{or} \quad \frac{d^n f(x)}{dx^n} \quad \text{or} \quad
y^{(n)}(x) \quad \text{or} \quad \frac{d^ny(x)}{dx^n} \quad \text{or} \quad \left( \frac d {dx} \right) ^n f(x)
\end{align*}
$\frac d {dx}$ is called the differential operator.


\begin{ex}
\begin{align*}
& f(x) = \log x & & f^{(1)}(x) = \frac 1 x \\
& f^{(2)} (x) = -\frac 1 {x^2} & & f^{(3)}(x) = \frac 2 {x^3} \\
& f^{(4)} (x) = -\frac{2 \cdot 3}{n^4} & & f^{(n)} (x) = \frac{(-1)^{n+1} (n-1)!}{x^n}
\end{align*}
\end{ex}

\begin{tm}
The Leibniz' formula is
\begin{align*}
\left(\frac d {dx} \right) u v = \sum_{p=0}^n \binom n p u^{n-p} v^p
\end{align*}
\end{tm}
The derivation can be made through regarding the functions $u(x)v(x)$.
\begin{align*}
\frac d {dx} u v = u'v + uv'
\end{align*}
Differentiating again gives us
\begin{align*}
\left( \frac d {dx} \right)^2 uv & = u'' v + u'v' + u'v' + uv'' \\
& = u''v + 2u'v' + u v'' \\
\left( \frac d {dx} \right)^3 uv & = u''v + u''v' + 2(u''v'+ u'v'') + u'v'' + uv''' \\
& = u'''v + 3u''v' + 3u'v'' + uv'''
\end{align*}
The coefficients are binomial coefficients. A rigorous proof can be made by induction.



\begin{ex}
\begin{itemize}
\item
Leibniz is particularly useful if one term in the product is a polynomial -- since the sum terminates
\begin{align*}
f(x) & = e^{2x}x^2 
\end{align*}
Set
\begin{align*}
v= x^2, \quad u = e^{2x} 
\end{align*}
Then
\begin{align*}
v^{(1)} = 2x, \quad v^{(2)} = 2, \quad v^{(3)} = v^{(4)} = v^{(5)} & = 0 \\
u^{(n)} & = 2^n e^{2x}
\end{align*}
This gives us the $n^{th}$ derivative of $f$
\begin{align*}
f^{(n)} (x) & = \binom{n}{0} u^{(n)} v^{(0)} + \binom{n}{1} u^{(n-2)} v^{(1)} \binom{n}{2} u^{(n-2)} v^{(2)} \\ 
& = 2^n e^{2x}  x^2 + n 2^{n-1} e^{2x} 2x   + n(n-1)2^{n-2} e^{2x} 
\end{align*}
\item
Another example
\begin{align*}
f(x) & = \sin^{-1} x \\
f'(x) & = \frac 1 {\sqrt{1-x^2}} \\
f''(x) & = \frac x {(1-x^2)^{\frac 3 2}} = \frac x {1-x^2} f'(x) \\
(1-x^2)f^{(2)}(x) & = x f^{(1)} (x) 
\end{align*}
Differentiate both sides $n$ times.
\begin{align*}
\left( 1-x^2 \right) f^{(2+n)} + \binom n 1 (-2x) f ^{(1+n)} + \binom n 2 (-2) f^{(n)} = xf^{(n+1)} + 1 f^{(n)} \binom{n} 1 \\
\left( 1-x^2 \right) f^{2+n} - 2nx f(1+n)(x) - n (n+1) f^{(n)} = x f^{(n+1)} +  nf^{(n)}
\end{align*}
Set $x=0$
\begin{align*}
f^{(2+n)}(0) - n(n-1)f^{(n)}(0) & = n f^{(n)}(0)  \\
f^{(2+n)}(0) & = n^2 f^{(n)}(0) \\
f^{(0)}(0) & = 0 \\
f^{(1)}(0) & = 1 \\
f^{(3)}(0) & = 1 \\
f^{(5)}(0) & = 9 = 3^2 \\
f^{(7)}(0) & = 3^2 5^2 = 225 \\
f^{(9)}(0) & = 3^2 5^2 7^2
\end{align*}
\end{itemize}
\end{ex}



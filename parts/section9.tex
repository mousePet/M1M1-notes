\section{ODEs}

\begin{df}
	An \emph{ordinary differential equation} (ODE) is an equation involving a function $y$ and one or more of its derivatives.
\end{df}

The general form is
\begin{align*}
h(y,y',y'', \dots y^{(n)}, x) = 0
\end{align*}
This has order $n$ where $n$ is the highest derivative present. There are linear and non-linear ODEs.

\begin{df}
	A \emph{linear ODE} can be written as
	\begin{align*}
	p_0(x) y + p_1(x) y' + \dots + p_n(x) y^{(n)} = q(x)
	\end{align*}
	where $p_0, p_1, \dots, p_n$ are functions of $x$.
\end{df}

\begin{nex}
	\item
	A first order linear equation is
	\begin{align*}
	y' = \alpha y \qquad \alpha \in \R.
	\end{align*}
	\item
	A first order non-linear equation is
	\begin{align*}
	y' = \alpha y^2.
	\end{align*}
	\item
	The following equation is second order and non-linear
	\begin{align*}
	y'' = -1 -(y')^2
	\end{align*}
	\item
	A second order linear equation is
	\begin{align*}
	x^2 y'' + xy' + y = \tan(e^x)
	\end{align*}
\end{nex}



\subsection{First order ODEs}
The general form is 
\begin{align*}
h(y,y',x) = 0.
\end{align*}
The simplest possible 1st order ODE is
\begin{align*}
y'(x) = f(x)
\end{align*}
The solution is
\begin{align*}
y(x9 = \int f(x) ~dx.
\end{align*}
\begin{ex}
	\begin{align*}
	\frac{dy}{dx} = \frac 1 {\sqrt{1-x^2}}
	\end{align*}
	has the general solution 
	\begin{align*}
	y(x) = \sin^{-1} x + C
	\end{align*}
\end{ex}

The general linear ODE of order one looks like
\begin{align*}
y'(x) + p(x) y(x) = q(x)
\end{align*}
where $p$ and $q$ are arbitrary functions.
(Integrating the equation does not always work:
\begin{align*}
y(x) + \int p(x) y(x)~dx = \int q(x)~dx
\end{align*}
)

The integrating factor method helps. Multiply the ODE by 
\begin{align*}
e^{\int p(x)~dx}
\end{align*}
We get
\begin{align*}
e^{\int p(x) ~dx} (y'(x) + p(x) y(x)) = q(x) e^{\int p(x) ~dx}
\end{align*}
The LHS can be written as
\begin{align*}
\frac d {dx} \left( y(x) e^{\int p(x)~dx} \right) = q(x) e^{\int p(x)~dx} \\
\frac d {dx} e^{\int p(x)~dx} = p(x) e^{\int p(x)~dx}
\end{align*}
Now directly integrate 
\begin{align*}
y(x) e^{\int p(x)~dx} = \int q(x) e^{\int p(x)~dx} ~dx \\
y(x) = e^{-\int p(x)~dx} \cdot \int q(x) e^{\int p(x)~dx} ~dx
\end{align*}
\begin{ex}
	\begin{align*}
	y' + xy = x \qquad p(x) = x, q(x) = x
	\end{align*}
	The integrating factor is 
	\begin{align*}
	e^{\int p(x)~dx} = e^{\int x~dx} = e^{\frac 1 2 x^2} 
	\end{align*}
	We multiply the PDE by $e^{\frac 1 2 x^2}$.
	\begin{align*}
	e^{\frac 1 2 x^2} (y'+xy) & = xe^{\frac 1 2 x^2} \\
	\frac d {dx} \left(y e^{\frac 1 2 x^2} \right) & = xe^{\frac 1 2 x^2}
	\end{align*}
	We integrate both sides
	\begin{align*}
	y e^{\frac 1 2 x^2} = \int x e^{\frac 1 2 x^2}~dx \\
	= e^{\frac 1 2 x^2} +C \\
	y = e^{-\frac 1 2 x^2}\left( e^{\frac 1 2 x^2} +C\right) \\
	= 1 + C e^{-\frac 1 2 x^2}
	\end{align*}
	where $C$ is an arbitrary constant.
\end{ex}


\subsection{Separation of Variables}
We can solve certain linear and non-linear ODEs 
\begin{nex}
	\item
	\begin{itemal}
	\frac{dy}{dx} = y \sin x  
	\end{itemal}
	Separate the $x$ and $y$ dependence.
	\begin{align*}
	\frac{dy} y = \int \sin x ~dx \\
	\log y = - \cos x +C \\
	y = Ae^{-\cos x}
	\end{align*}
	where $A$ is an arbitrary constant.
	\item
	\begin{itemal}
		\frac{dy}{dx} = y^2 \sin x \\
		\int \frac{dy}{y^2} = \int \sin x~ dx \\
		- \frac 1 y = - \cos x +C \\
		\frac 1 y = \cos x + d \\
		y = \frac 1 {d+\cos x}
	\end{itemal}
	where $d$ is an arbitrary constant.
\end{nex}

\subsection{Homogeneous ODEs}
\begin{df}
	A \emph{homogeneous ODE} of first order is something like
	\begin{align*}
	\frac{dy}{dx} = f\left( \frac y x\right)
	\end{align*}
	where $f$ is an arbitrary function.
\end{df}

\begin{ex}
	\begin{align*}
	\frac{dy}{dx} = \frac y x - \tan \left( \frac y x \right)
	\end{align*}
	Use a new function $v = \frac y x$ and work with that instead of $y$.
	\begin{align*}
	y= xv \\
	y' = xv' + v 
	\end{align*}
	So we can write the ODE as 
	\begin{align*}
	xv' + v = f(v)
	\end{align*}
	which is always separable. 
	\begin{align*}
	v' = \frac{f(v) - v} x
	\end{align*} 
	
	\begin{align*}
	xv' + v = v - \tan v \\
	x \frac{dv}{dx} = -\tan v \\
	\frac{dv}{\tan v} = - \frac{dx} x \\
	\int \frac{\cos v ~dv}{\sin v} = -\int \frac {dx} x \\
	\log(\sin v) = - \log x + C \\
	\sin v = \frac A x
	\end{align*}
	where $A$ is an arbitrary constant.
\end{ex}


Bernoulli (non-linear)
\begin{align*}
y' + py = qy^n
\end{align*}
Use the trick substitution $z = y^{1-n}$
\begin{align*}
z' = (1-n) y^{-n} y'
\end{align*}
multiply equation by $(1-n)y^{-n}$:
\begin{align*}
(1-n)y^{-n} y' + p(1-n)y^{1-n} = q(1-n) \\
z' + (1-n)pz = q(1-n)
\end{align*}
which is linear.

\subsection{Second Order equations}
\begin{df}
A general second order linear equation is
\begin{align*}
y'' + p(x) y' + q(x) y = r(x)
\end{align*} 
where $p,q$, and $r$ are arbitrary functions of $x$.
\end{df}

There is no general solution known. The form of a general solution is
\begin{align*}
y(x) = C_1 y_1(x) + C_2y_2(x) + y_{PI} (x)
\end{align*}
were $C_1$ and $C_2$ are arbitrary constants. $y_1(x)$ and $y_2(x)$ are independent solutions of 
\begin{align*}
y'' + p(x) y' + q(x) y = 0
\end{align*}
($r$ is set to zero -- called homogeneous form of ODE)
independent means that $y_1(x)$ $y_2(x)$ are not proportional to each other.
$y_{PI}(x)$ is one solution of the full ODE. 

In general it is difficult to find $y_1(x)$ and $y_2(x)$. However, special cases can be solved. A useful special case is the one with constant coefficients. ($p$ and $q$ but not necessarily $r(x)$ are constant.)
The constant coefficient case is usually written
\begin{align*}
ay'' + by' + cy = r(x)
\end{align*}
where $a,b,c$ are constants. this is easy to solve (the solution is usually exponential). $y= e^{\lambda x}$ will work for some constant $\lambda$. The homogeneous equation
\begin{align*}
ay'' + by' + cy =0 \\
y = e^{\lambda x} \\
y' = \lambda e^{\lambda x} \qquad y''= \lambda^2 e^{\lambda x} \\
\Rightarrow \qquad a \lambda^2 + b \lambda + c = 0 
\end{align*}
quadratic equation for $\lambda$.
\begin{align*}
y_1(x) = e^{\lambda_1 x} \qquad \lambda_2(x) = e^{\lambda_2}
\end{align*}
auxialiary equation
\begin{align*}
a\lambda^2+b \lambda + c = 0
\end{align*}
(if $\lambda_1 = \lambda_2$ see next term)
hard bit to find $y_{PI}(x)$ one solution of full equation.
\begin{align*}
y(x) C_2 e^{\lambda_1 x} + C_2 e^{lambda_2 x} + y_{PI} (x)
\end{align*}
An important application of ODEs is:
\begin{itemize}
	\item 
	Mechanics \\
	Motion in one dimension. \\
	Newton's second law (find position as a function of time $t$)
	\begin{align*}
	ax \dot \dot \dot x = F(x, \dot x, t)
	\end{align*}
	\begin{ex}
		A boulder falls from a cliff.
		$z$ measures the height above the cliff top -- $z$ is negative during the fall
		\begin{align*}
		m \dot \dot z = - mq+\alpha (\dot z)^2
		\end{align*}
		ODE 
		\begin{align*}
		\dot \dot z = -q + \beta \dot z^2
		\end{align*}
		$\beta = \frac \alpha m =$ constant. This is second order but first order if we substitute $w=\dot z$ in $w$.
		\begin{align*}
			\dot w = -g + \beta w^2 \\
			\int \frac{dw}{g-\beta w^2} = - \int df \\
			\int \frac{df}{1-x^2} = \tanh^{-1} x +C \\
			\frac 1 g \int \frac{dw}{1- (\sqrt{\frac \beta g} w)^2} = -dt \\
			\frac 1 g \sqrt{\frac g \beta} \tanh^{-1}(\sqrt{\frac \beta g} w) = -t +c \\
			= \frac 1 {\sqrt{\beta g}} \tanh^{-1}(\sqrt{\frac \beta g} w) = -t+C
		\end{align*}
		At $t=0, z=0, \dot z = 0$. The boulder has zero speed initially. $\rightarrow c=0$.
		\begin{align*}
			\tanh^{-1} (\sqrt{\frac \beta g} w) = - \sqrt{\beta g} t \\
			\sqrt{\frac \beta g} w = - \tanh(\sqrt{\beta g} t) \\
			w(t) = - \sqrt{\frac g \beta} \tanh(\sqrt {\beta g} t )
		\end{align*}
		If $\beta$ is small, i.e. the air resistance is small then
		\begin{align*}
			\tanh(\sqrt{\beta g} t) \approx \sqrt{\beta g} t \\
			\tanh x \approx x - \frac{x^3}{3} + \dots \\
			w(t) \approx - gt \\
			z = - \frac 1 2 g t^2
		\end{align*}
		For large $t$
		\begin{align*}
			w(t) \rightarrow - \sqrt{\frac g \beta}
		\end{align*}
		constant terminal velocity.
		To work out $z(t)$ we have to integrate $w(t)$.
	\end{ex}
\end{itemize}






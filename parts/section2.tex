\section{Limits}
 \begin{df}
  The symbolic notation
  \begin{align*}
  L = \lim_{x \to a} f(x) \quad \operatorname{or} \quad f(x) \overset{x \to a}{\to} f(x)
  \end{align*}
  means:
 \begin{align*}
  \forall \epsilon > 0 ~\exists \delta>0: ~(|x-a|> \delta \wedge x \in \dom(f)) \Rightarrow |f(x) - L| < \epsilon
 \end{align*}
 $f(x)$ approaches $L$ as $x$ approaches $a$. 
 \end{df}
\begin{rk}
It is important that $x$ approaches $a$ from both the left and the right.
A one sided limit is written as follows:
\begin{align*}
\lim_{a \to a^+} f(x)  
\end{align*}
or
\begin{align*}
 \lim_{x \to a^-} f(x)
\end{align*}
\end{rk}


\begin{ex} \mbox \\
\begin{itemize}
\item
Let
\begin{align*}
f(x) & = \frac x {|x|}, & x \neq 0
\end{align*}
Then
\[
\lim_{x \to 0} f(x)
\]
is undefined but one sided limits exist:
\begin{align*}
\lim_{x \to 0^+} f(x) & = 1 \\
\lim_{x \to 0^-} f(x) & = -1 
\end{align*}
\item Let
\begin{align*}
f(x) = x^2 
\end{align*}
Then
\begin{align*}
\lim_{x \to 2} f(x) = f(2) = 4
\end{align*}
\item
Let
\begin{align*}
f(x) & = \frac{x^2-1}{x-1} 
\end{align*}
$\lim_{x_ \to 1} f(x)$ is an indeterminate limit of the form $\frac 0 0$. However $\frac{x^2-1}{x-1} = x+1$ if $x \neq 1$. Therefore $\lim_{x_ \to 1} f(x) = 2$.
\end{itemize}
\end{ex}

\begin{rk}
Not all indeterminate limits are meaningful.
\end{rk}

\begin{ex}
The limit
\begin{align*}
\lim_{x \to 1} \frac{x^2-1}{(x-1)^2}
\end{align*}
of the form $\frac 0 0$ does not exist.
\end{ex}

\subsection{Infinite Limits}
\begin{ex}
\begin{align*}
\lim_{x \to \infty} \frac 1 x & = 0 \\
\lim_{x \to -\infty} \tan^{-1} x & = -\frac \pi 2 \\
\end{align*}
$x \to \infty$ is the same as $\frac 1 x \to 0^+$. \\
$x \to -\infty$ is the same as $\frac 1 x \to 0^-$.
\end{ex}
\begin{pr} 
Provided the limits $\lim_{x \to a} f(x)$ and $\lim_{x \to a} g(x)$ exist we know the following rules:
\begin{itemize}
\item
addition formula
\begin{align*}
 \lim_{x \to a} (f(x) +g(x) ) & = \lim_{x \to a} f(x) + \lim_{x \to a} g(x)
\end{align*}
\item 
product rule
\begin{align*}
\lim_{x \to a} f(x) g(x) = \lim_{x \to a} f(x) \cdot \lim_{x \to a} g(x)
\end{align*}
\item quotient rule ($\lim_{x \to a} g(x) \neq 0$)
\begin{align*}
\lim_{x \to a} \frac{f(x)}{g(x)} = \frac{\lim_{x \to a} f(x)}{\lim_{x \to a} g(x)}
\end{align*}
\end{itemize}
\end{pr}

\subsection{Computing Limits}
\begin{itemize}
\item 
manipulate function so the limit is 'obvious'
\item
use power-series
\item
L'Hopital's rule
\end{itemize}

\begin{ex} \mbox \\
\begin{itemize}
\item
\begin{align*}
L & = \lim_{x \to 1} \frac{\sqrt{2-x} -1}{1-x} \\
 & = \lim_{x \to 1} \frac{\sqrt{2-x}-1}{1-x} \cdot \frac{\sqrt{2-x}-1}{\sqrt{2-x}-1} \\
& = \lim_{x \to 1} \frac{(2-x) -1}{(1-x)(\sqrt{2-x}+1} \\
& = \lim_{x \to 1} \frac{1-x}{(1-x)\left(\sqrt{2-x}+1\right)} \\ 
& = \lim_{x \to 1} \frac 1 {\sqrt{2-x} +1} \\
 & = \frac 1 2
\end{align*}
Alternatively the power series can be used. Let $s= 1-x$. Then
\begin{align*}
L & = \lim_{s \to 0} \frac{\sqrt{1+s}-1} s
\end{align*}
\begin{tm}
The general binomial theorem says
\begin{align*}
(1+s)^p = 1+ps + \frac{p(p-1)}{2!} s^2 + \frac{p(p-1)(p-2)}{3!} s^3 + \dots
\end{align*}
for $|s|<1$.
\end{tm}
The geometric series is a special case of the binomial theorem with $p=-1$.

If $p$ is a positive integer, the series terminates -- and gives us the standard binomial theorem. If $p$ is not a positive integer, the formula continues infinitely. Hence 
\begin{align*}
(1+s)^{\frac 1 2} = 1 + \frac 1 2 s + \frac{\frac 1 2 ( \frac 1 2 -1)}{2!} s^2 + \dots
\end{align*}
can be inserted in our formula and we get:
\begin{align*}
\lim_{s \to 0} \frac{\sqrt{1+s}-1}{s} = \lim_{s \to 0} \frac{1 + \frac 1 2 s + \frac{\frac 1 2 \left( \frac 1 2 -1 \right)}{2!} s^2 + \dots -1}{s} =  \frac 1 2
\end{align*}
\item Let us calculate the following well-known limit:
\begin{align*}
\lim_{x\to 0} \frac{\sin x}{x}
\end{align*}
Using a power series we get
\begin{align*}
\lim_{x\to 0} \frac{\sin x}{x} = \lim_{x\to 0} \left(1-\frac{x^2}{3!} + \frac{x^4}{5!} - \dots\right) = 1
\end{align*}
\item Having calculated this limit we can compute
\begin{align*}
\lim_{x \to 0} \frac{\tan x}{x} = \lim_{x \to 0} \frac{\sin x}{x} \cdot \cos x = 1
\end{align*}
\item
Using limits for graph sketching.
\begin{align*}
f(x) = \frac x {e^x-1}
& = \frac 1 {1 + \frac x {2!} + \frac{x^2}{3!} + \dots}
 \overset{x\to 0}{\to} 1 
\end{align*}
\item Let
\begin{align*}
f(x) = \frac{\cos \left( \frac \pi 2 x \right) }{ 1-x^2}
\end{align*}
Now we can calculate $\lim_{x \to 1} f(x)$.
\begin{align*}
& \frac{\cos \left( \frac \pi 2 x \right) }{ 1-x^2} \\ 
= & \frac{\cos\left( \frac \pi 2 (x-1) + \frac \pi 2\right)}{ (x-1)(x+1)} \\
= & \frac{\sin\left(\frac \pi 2(x-1) \right) }{(x-1)(x+1)}
\end{align*}
Substituting $s=x-1$ that gives us
\begin{align*}
= & \frac{\sin \left(\frac \pi 2 s\right)}{s(2+s)} \\
= & \frac{\frac \pi 2 s - \frac 1 {3!} \left( \frac \pi 2 s \right)^3 + \frac 1 {5!} \left( \frac \pi 2 s \right)^5 + \dots}{s(2+s)} \\
= & \frac{\frac \pi 2 - \frac 1 {3!} \left( \frac \pi 2 s \right)^3 + \frac 1 {5!} \left( \frac \pi 2 s \right)^5 + \dots}{2+s} \\
 \overset{s \to 0}{\to} & \frac \pi 4 
\end{align*}
\item
Consider the limit
\begin{align*}
& \lim_{x\to \infty} x^{\frac 1 3} \left( (x+1)^{\frac 2 3} - x^{\frac 2 3} \right) \\
= & \lim_{x\to \infty} x^{\frac 1 3} \left( x^{\frac 2 3} (1+ \frac 1 x)^{\frac 2 3} - x^{\frac 2 3} \right) \\
= & \lim_{x\to \infty} x^{\frac 1 3} \left( x^{\frac 2 3} \left( 1+ \frac 2 3 \cdot \frac 1 x + \frac{\frac 2 3 \left( \frac 2 3 -1 \right)}{2!} \left( \frac{1}{x} \right)^2 + \dots \right) - x^{\frac 2 3} \right) \\
= & \lim_{x\to \infty} x \left( 1+ \frac 2 3 \cdot \frac 1 x + \frac{\frac 2 3 \left( \frac 2 3 -1 \right)}{2!} \cdot \left( \frac{1}{x} \right)^2 + \dots \right) - x \\
= & \lim_{x\to \infty} \left( x \left( 1+ \frac 2 3 \cdot \frac 1 x + \frac{\frac 2 3 \left( \frac 2 3 -1 \right)}{2!} \cdot \left( \frac{1}{x} \right)^2 + \dots \right) - x \right) \\
= & \lim_{x\to \infty} \left( x + \frac 2 3 + \frac{\frac 2 3 \left( \frac 2 3 -1 \right)}{2!} \cdot \frac{1}{x} + \dots  - x \right) \\
= & \frac 2 3
\end{align*}
\item 
Using limits, the equivalence of the definitions for the exponential function can be proven.
\begin{align*}
\lim_{x \to \infty} \left(1+ \frac a x \right)^x = e^a
\end{align*} 
Derivation:
\begin{align*}
& \lim_{x \to \infty} \left( 1+\frac a x \right)^x \\
= & \lim_{x \to \infty} \exp \left( \log \left( 1+ \frac a x \right)^x \right) \\
= & \lim_{x \to \infty} \exp \left(x \log \left( 1+ \frac a x \right) \right) \\
= & \exp \left( \lim_{x \to \infty} x \left( \frac a x- \frac 1 2 \cdot \left( \frac a x \right)^2 + \frac 1 3  \left(\frac a x \right)^3 - \frac 1 4  \left( \frac a x \right)^4 \dots \right) \right) \\
= & \exp (a)
\end{align*}
Another possibility to derive this is
\begin{align*}
\left( 1+\frac a x \right)^x
= & 1+ x \frac a x + \frac{x(x-1)}{2!} \left( \frac a x \right) ^2 \\
& + \frac{x(x-1)(x-2)}{3!} \left( \frac a x \right)^3 + \dots 
\end{align*}
Considering the limit to infinity and therefore only the dominant powers in each fraction we get
\begin{align*}
= 1+a+ \frac {a^2}{2}  +\frac{a^3}{3!} +\dots 
\end{align*}
\end{itemize}
\end{ex}







\subsection{Continuity}
\emph{Informal Definition.} A continuous function $f$ has a graph with no breaks or jumps.
\begin{ex} A continuous function is
\begin{align*}
f(x) = x^2
\end{align*}
An example for a non-continuous function is the Heaviside function:
\begin{align*}
H(x) = \left\{
\begin{array}{l l}
0, & x<0\\
1, & x\ge 0
\end{array}\right.
\end{align*} 
\end{ex}
\begin{df}
A function $f$ is continuous at $a \in \dom(f)$ if 
\begin{align*}
\lim_{x\to a} f(x) = f(a)
\end{align*}
\end{df}
As for the Heaviside function, $\lim_{x\to 0} H(x)$ doesn't exist. Nonetheless, $H(x)$ is continuous for all $x \neq 0$ .

\begin{ex}
\begin{align*}
f(x) = x \sin \left( \frac 1 x \right)
\end{align*}
is continuous for $x \neq 0$ but not continuous at $x=0$ (because it is not defined there). However,
\begin{align*}
\lim_{x\to 0} f(x) = 0
\end{align*}
since 
\begin{align*}
-|x| \leq f(x) \leq |x|
\left| \sin \frac  1 x \right| \leq 1
\end{align*}
Hence, we can consider the function
\begin{align*}
g(x) = x  \left\{ \begin{array}{ll}
\sin \left( \frac 1 x \right), & x \neq 0 \\
 0, & x=0 \\
\end{array} \right.
\end{align*} 
$g$ is continuous for all $x \in \R$.
\end{ex}

\subsection{List of Power Series}
\begin{align*}
\sin x & = x -\frac{x^3}{3!} + \frac{x^5}{5!} - \frac{x^7}{7!} + \dots \\
\cos x & = 1 -\frac{x^2}{2!} + \frac{x^4}{4!} - \frac{x^6}{6!} + \dots \\
\tan x & = x + \frac{x^3}{3} + \frac{2}{15}x^5 + \dots \\
\tan^{-1} x & = x + \frac{x^3}{3} + \frac{x^5}{5} - \frac{x^7}{7} + \dots \\
\log (1+x) & = x -\frac{x^2}{2} - \frac{x^3}{3} - \frac{x^4}{4} + \dots \\
e^x & = 1 + x + \frac{x^2}{2!} + \frac{x^3}{3!} + \dots
\end{align*}
To derive the power series for $\tan^{-1}$, consider
\begin{align*}
\frac 1 {1+x^2} = 1-x^2 +x^4-x^6 + \dots
\end{align*}
Integrating both sides gives us
\begin{align*}
\tan^{-1} x = x -\frac {x^3} 3 + \frac{x^5} 5 - \dots
\end{align*}


\section{Complex numbers}

\begin{df}
	A complex number $z$ is of the form
	\begin{align*}
	z = x + iy
	\end{align*}
	where $x$ and $y$ are a pair of real numbers. $i$ is the imaginary unit with basic property $i^2 = -1$.
\end{df}

Geometrically a complex number is a point in the complex plane.

Abbreviations:
\begin{center}
\begin{tabularx}{.5\textwidth}{XX}
	\toprule
	Expression & Abbreviation \\
	\toprule
	$x+i0$ & $x$ \\
	$0+iy$ & $iy$ \\
	$0 + i0$ & 0 \\
	\bottomrule
\end{tabularx}
\end{center}

We can use polar coordinates:
\begin{align*}
x & = r \cos \theta \\
y & = r i \sin \theta
\end{align*}
$r$ is the distance from the origin and $\theta$ is the angle from the positive x-axis (anti-clockwise).


\begin{df}
	The \emph{modulus} $|z|$ of a complex number
	\begin{align*}
	z = x + iy = r \cos \theta + i \sin \theta
	\end{align*}
	is
	\begin{align*}
	|z| = \sqrt{x^2+ y^2} = r
	\end{align*}
\end{df}

\begin{df}
An \emph{argument} of $z$ is the polar angle $\theta$. This is ambiguous since replacing $\theta$ with $\theta \pm 2 \pi$ has no effect on $z$.
\end{df}

\begin{ex}
	\begin{align*}
	z = 0 + 1i = i \\
	|i| = 1 \\
	\operatorname{arg}(i) = \frac \pi 2
	\end{align*}
	or $\frac{5 \pi} 2$, $\frac{9 \pi}{2}$.
\end{ex}

$\Arg(z) = \theta$ denotes the principal value of the argument with $-\pi < \theta \le \pi$	.

\begin{rk}
	\begin{align*}
	|z_1 z_2| = |z_1| |z_2|
	\end{align*}
\end{rk}
The rules of algebra of complex numbers are the same as for reals (remember that $i^2 = -1$).

\begin{df}
	If $z = x + iy$, the \emph{reciprocal} is defined as 
	\begin{align*}
	\frac 1 z = \frac{x-iy}{x^2+y^2}.
	\end{align*}
\end{df}

\begin{rk}
	\begin{align*}
	\displaystyle z \cdot \frac 1 z = 1
	\end{align*}
\end{rk}

In polar coordinates the reciprocal can by obtained by replacing $r$ with $\frac 1 r$ and $\theta$ with $-\theta$. If $z = r \cos \theta + i  r \sin \theta$, then 
\begin{align*}
\frac 1 z = \frac 1 r ( \cos \theta - i \sin \theta).
\end{align*} 

\begin{ex}
	\begin{align*}
	z & = 1 + i \sqrt 3 = 2 \left( \cos \frac \pi 3 + i \sin \frac \pi 3 \right) \\
	\frac 1 z & = \frac{1-i \sqrt 3}{1+3} \\
	& =  \frac{1-i \sqrt 3}{4} \\
	& = \frac 1 2 \left( \cos \frac \pi 3  - \sin \frac \pi 3 \right)
	\end{align*}
\end{ex}

Integer powers can be defined by multiplication
\begin{align*}
z^2 & = z \cdot z \quad z^3 = z \cdot z \cdot z \quad \dots \\
z^2 & = (x+iy)(x+iy) \\
& = x^2 - y^2 + 2ixy
\end{align*}
Negative integer powers are defined through the reciprocal.
\begin{align*}
z^{-1} & = \frac 1 z \\
z^{-2} & = \frac 1 z \cdot \frac 1 z \quad \\
\dots &
\end{align*}


\subsection{Power series}
\begin{df}
	A \emph{complex power series} has the form
	\begin{align*}
	\sum_{m=0}^\infty c_m z^m, \qquad c_i \in \C,~ i \in \N.
	\end{align*} 
\end{df}

The ratio test, the root test and the radius of convergence still work, except that instead of absolute values the complex modulus is used. 

\begin{ex}
	An important example is the complex exponential function. Recall the real exponential has a power series expansion
	\begin{align*}
	\exp(x) = 1 + \frac x {1!} + \frac{x^2}{2!} + \frac{x^3}{3!} + \dots
	\end{align*}
	with $R=0$.
	We can define the complex exponential
	\begin{align*}
	\exp(z) = 1 + \frac z {1!} + \frac{z^2}{2!} + \frac{z^3}{3!} + \dots
	\end{align*}
	with $R=0$. The right-hand side is convergent for any complex number $z$.
\end{ex}


The addition formula for the complex exponential is still true:
\begin{align*}
\exp(z+w) = \exp(z) \exp(w)
\end{align*}


\begin{tm}
	\emph{Euler's Formula} 
	\begin{align*}
	e^{i \theta} = \cos \theta + i \sin \theta 
	\end{align*}
\end{tm}

\begin{proof}
	Insert $z = i \theta$ into the exponential power series:
	\begin{align*}
	\exp(i \theta) & = 1 +  \frac{(i \theta)^2}{2!} + \frac{(i \theta)^4}{4!} + \dots + \frac {i \theta} {1!} + \frac{(i \theta)^3}{3!} + \frac{(i \theta)^5}{5!} + \dots \\
	& = 1 -  \frac{\theta^2}{2!} + \frac{ \theta^4}{4!} + \dots + i \left(\frac {i \theta} {1!} - \frac{ \theta^3}{3!} + \frac{ \theta^5}{5!} + \dots \right) \\
	& = \cos \theta + i \sin \theta
	\end{align*}
\end{proof}

Hence, we can rewrite the polar form of $z = r \cos \theta + i r \sin \theta$ as $z = r e^{i \theta}$

\begin{ex}
	\begin{itemize}
		\item
	\begin{align*}
	\sqrt 3 + i = 2\left( \cos \frac \pi 6 + i \sin \pi 6 \right) = e^{\frac{i \pi} 6}
	\end{align*}
	\item
	\begin{align*}
	& &	z & = -1 \\
	& &	 |z| & = 1, \quad \arg(z) = \pi \\
		& \Rightarrow & -1 & = e^{i\pi}  \\
	& \Leftrightarrow &	1 + e^{i\pi} & = 0
	\end{align*}
\end{itemize}
\end{ex}


We can use Euler to derive trigonometrical formulas. For example the addition formulas:
\begin{align*}
\sin( \alpha + \beta) & = \sin \alpha \cos \beta  + \sin \beta \cos \alpha \\
\cos( \alpha + \beta) & = \cos \alpha \cos \alpha  + \cos \beta \cos \beta
\end{align*}
\begin{proof}
	Apply Euler to the three exponentials.
	\begin{align*}
	& \cos(\alpha+ \beta)  + i \sin (\alpha+ \beta) \\
	= & (\cos \alpha + i \sin \alpha) (\cos \beta + i \sin \beta) \\
	= & \cos \alpha \cos \beta - \sin \alpha \sin \beta + i( \cos \alpha \sin \beta + \sin \alpha \cos \beta)
	\end{align*}
	The real part is the addition formula for cosine and the imaginary part gives us the addition formula for the sine function.
\end{proof}

\begin{tm}
\emph{De Moivre's Theorem} 
\begin{align*}
(\cos \theta + i \sin \theta )^n = \cos n \theta + i \sin n \theta
\end{align*}
\end{tm}

\begin{proof}
	Apply Euler to two exponentials $e^{i \theta}$ and $e^{in \theta}$.
\end{proof}

Euler's formula expresses a complex exponential in terms of trigonometrical functions. We can also do the reverse, i.e. express trigonometrical functions in terms of complex exponentials.
\begin{align*}
\cos \theta & = \frac{e^{i \theta} + e^{-i \theta}} 2 \\
\sin \theta & = \frac{e^{i \theta} + e^{-i \theta}} {2i} 
\end{align*}
The derivation can be made using Euler and 
\begin{align*}
e^{- i \theta} = \cos \theta - i \sin \theta
\end{align*}
We can generalise cosine and sine to complex arguments.
\begin{df}
	For any complex number $z$ 
\begin{align*}
\cos z = \frac{e^{iz} + e^{-iz}} 2 \\
\sin z = \frac{e^{iz} + e^{-iz}} {2i}.
\end{align*}
\end{df}

\begin{ex}
	\begin{align*}
	\sin i = \frac{e^{i^2} - e^{-i^2}}{2i} = \frac {e^{-1} -e}{2i}
	\end{align*}
\end{ex}

\begin{df}
The complex conjugate of a complex number
\begin{align*}
z = x + iy = r e^{i \theta}
\end{align*}
is defined as
\begin{align*}
\overline z = x-iy = r e^{-i\theta}
\end{align*}
\end{df}

Geometrically, complex conjugation is a reflection about the real axis.

Useful formulas are
\begin{align*}
\operatorname{Re}(z) & = \frac{z + \overline z} 2 \\
\operatorname{Im}(z) & = \frac{z- \overline z} 2 \\
|z| & = (z \overline z)^{\frac 1 2} \\ \
\overline{ab} & = \overline a\cdot  \overline b
\end{align*}

\subsection{Complex Polynomials}

\begin{df}
A complex Polynomial of degree $n$ has the form
\begin{align*}
P(z) = c_0 + c_1 z + c_2 z^2 + \dots + c_n z^n, \qquad c_0, c_1, \dots, c_n \in \C, c_n \neq 0.
\end{align*}
\end{df}


\begin{tm}
	\emph{Fundamental Theorem of Algebra} \\
	A complex polynomial has at least one root.
\end{tm}

A polynomial of degree $n$ has therefore $n$ roots, (where roots may be repeated). A complex polynomial can be factorized into linear factors.
I.e. we can write
\begin{align*}
P(z) = c_n(z-a_1)(z-a_2)\dots (z-a_n)
\end{align*}
where $a_1, a_2, \dots, a_n$ are the $n$ roots of $P$ (which can be repeated, e.g.. $a_1=a_2$).

If the coefficients $c_0, c_1, \dots, c_n$ of the polynomial are real, the roots can still be complex. 
\begin{pp}
	If the coefficients are real, the roots are either real or appear in complex conjugate pairs.
\end{pp}

\begin{proof}
	Suppose $a$ is a root of $P$, i.e.
	\begin{align*}
	P(a) = c_0 + c_1 a + c_2a^2 + \dots + c_n a^n = 0.
	\end{align*}
	We take the complex conjugate
	\begin{align*}
	\overline c_0 + \overline c_1 \overline a + \overline c_2\overline a^2 + \dots + \overline c_n \overline a^n = 0.
	\end{align*}
	However, we assumed that the coefficients are real. Hence
	\begin{align*}
	 c_0 +  c_1 \overline a +  c_2\overline a^2 + \dots +  c_n \overline a^n = P(\overline z) = 0.
	\end{align*}
\end{proof}

\begin{ex}
	\begin{itemize}
		\item
	$ \displaystyle
	P(z) = z^6 + 7z^3 -8
	$ \\
	has six complex roots.
	\begin{align*}
	P(z) & = (z^3)^2 + 7z^3 -8 \\
	& = (z^3+8)(z^3-1)
	\end{align*}
	Hence, the roots are solutions of $z^3-1=0$ or $z^3+8 =0$. Let us first consider the former:
	\begin{align*}
	& & z^3 & = 1 = e^{2\pi i} \\
	& \Leftrightarrow & z =1 \quad \vee \quad z & = e^{\frac{2\pi i}{3}} \quad \vee \quad  z = e^{\frac{4\pi i} 3} = e^{\frac{-2\pi i}{3}}
	\end{align*}
	We can do the same with the other equation:
	\begin{align*}
	& & z^3 & = -8 = 8 e^{\pi i} \\
	& \Leftrightarrow & z = -2 \quad \vee \quad z & = 2 e^{\frac{\pi i}{3}} \quad \vee \quad  z = e^{\frac{5\pi i} 3} = e^{\frac{-\pi i}{3}}
	\end{align*}
	As a result, we can factorize $P(z)$ as
	\begin{align*}
	(z-1)(z-2)(z-e^{\frac{2\pi i}{3}})(z-e^{\frac{-2\pi i}{3}}) (z-2e^{\frac{\pi i}{3}})(z-2e^{\frac{-\pi i}{3}})
	\end{align*}
	\item
	Similarly, we solve the following equation:
	\begin{align*}
	z^3 -i & = 0 \\
	z^3 & = i = e^{i \frac \pi 2} = e^{5i \frac \pi 2} = e^{9i \frac \pi 2} \\
	z & = e^{i \frac \pi 6}, z = e^{5 i \frac \pi 6}, z = e^{9 i \frac \pi 6}
	\end{align*}
\end{itemize}
\end{ex}

\begin{df}
	A \emph{real Polynomial} of degree $n$ has the form
	\begin{align*}
	P(x) = c_0 + c_1 z + c_2 z^2 + \dots + c_n z^n, \qquad c_0, c_1, \dots, c_n \in \R, c_n \neq 0.
	\end{align*}
\end{df}

A real polynomial can be factorized into real linear and real quadratic factors. I.e. we can write
\begin{align*}
P(x) = c_n(x-a_1)(x-a_2) \dots (x-a_n)
\end{align*}
where $a_1,a_2, \dots, a_n$ are the roots of the complex polynomial
\begin{align*}
P(z) = c_0 + c_1 z + c_2 z^2 + \dots + c_n z^n.
\end{align*}
We can combine $c_1, c_2$ conjugate pairs into a real quadratic term.

\begin{ex}
	We already found out that the complex polynomial $P$ can be factorised in the following way:
	\begin{align*}
	P(z) &  = z^6 + 7z^3 - 8 \\
	& = (z^3+8)(z^3-1) \\
	& = (z-1)(z+2) \left(z - e^{2i \frac \pi 3} \right) \left(z - e^{-2i \frac \pi 3} \right) \left(z - e^{i \frac \pi 3} \right) \left(z - e^{-i \frac \pi 3} \right)
	\end{align*}
	Now, consider the real polynomial
	\begin{align*}
	P(x) & = z^6 + 7z^3 - 8 \\
	& = (x-1)(x+2) \left( x^2+ x\left( e^{2i \frac \pi 3} + e^{-2i \frac \pi 3} \right) +1 \right) \left( x^2 + x \left(e^{i \frac \pi 3} + e^{-i \frac \pi 3} \right) +4 \right) \\
	& = (x-1)(x-2)(x^2 +x+1) (x^2-2x+4)	.
	\end{align*}
\end{ex}

\subsection{Complex Functions}

\begin{df}
	A \emph{function $f$ of a complex variable} assigns a complex number $f(z)$ to every $z$ in $\dom (f) \subseteq \C$.
\end{df}

\begin{ex}
	\begin{itemize}
		\item
		complex polynomials
		\item
		complex power series
		\item
		complex exponential
		\item
		complex trigonometrical functions
		\item
		complex hyperbolic functions
	\end{itemize}
\end{ex}

\subsubsection{Complex Logarithm} 
\begin{df}
	Define $\log$ with
	\begin{align*}
	\exp(\log z) = z
	\end{align*}
	for $z \in \C$.
	We can use the polar form $z = r e^{i \theta}$ to simplify that:
	\begin{align*}
	z & = e^{\log r + i \theta} \\
	\log z & = \log r + i \theta \\
	\end{align*}
	The ambiguity can be omitted through
	\begin{align*}
	\log z = \log |z| + i \Arg (z).
	\end{align*}	
\end{df}

The complex logarithm has a singularity at zero and the "branch cut" of discontinuity at the negative real axis.

\subsubsection{Inverse Tangent Function} 
\begin{df}
	We define the \emph{inverse tangent function} through the power series
\begin{align*}
tan^{-1} u = z - \frac{z^2} 3 + \frac{z^5} 5 + \dots.
\end{align*}
\end{df}
In the complex plane the inverse tangent function has two branch cuts.
A nice way to see this is
\begin{align*}
\tanh^{-1} x & = \frac 1 2 \log \frac{1+x}{1-x} \\
\tan^{-1} x & = \frac 1 {2i} \log \frac{1+iz}{1-iz}.
\end{align*}
Since there are 2 logarithms involved there are two Branch cuts.

\subsubsection{Powers} 
\begin{df}
We define 
\begin{align*}
z^p = e^{p \log z}.
\end{align*}
\end{df}


\begin{align*}
\log z & = \log r + i \theta \\
z^p & = e^{p( \log r + i \theta)} \\
& = r^p e^{ip \theta}
\end{align*}
is ambiguous unless $p$ is an integer. Try
\begin{align*}
\log z^p = e^{p \log z}
\end{align*}
This again has a Branch cut on the negative real axis.
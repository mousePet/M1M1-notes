

%profressor Chris Ford
%c.ford@imperial.ac.uk
%651 huxley
%office hour 12 pm tuesday
%jonathan mastel google website 


\section{Functions}
\begin{df}
 A function $f$ is a rule assigning every element x in a set $A$ an element $f\left(  x \right)$ in another set $B$.
 \end{df}
 \begin{rk} \mbox \\
 \begin{itemize}
 \item
 $A$ is called the domain of $f$ whereas $B$ is called codomain.
 \item
 The range (image) of a function is the set:
 \begin{align*}
  \operatorname{Range(f)} & = \operatorname{Im} (f) \subseteq \operatorname{codomain} \\
  & = \{ f(x) \in B| ~\forall x \in A \}
 \end{align*}
 It does not have to be equal to the codomain.
 \item
 In the following we will mostly consider functions of one variable (with $A = \R$ and $B = \R$, later $\mathbb C$).
 \end{itemize}
 \end{rk}
 
 
 \begin{ex} Polynomials, $c_i \in \R, \forall i\in \mathbb N$:
 \begin{align*}
  f(x) = c_0 + c_1 x + c_2 x^2 + c_3 x^3 + \dots
  \end{align*}  
  \end{ex}
\begin{df} The graph of a function $f$ (real not complex) is the set 
\begin{align*}
\{\left( x,y\right)|x \in \dom( f), y = f(x)\}
\end{align*}
\end{df}
\begin{pr}
The graph of any function intersects any vertical line at most once.
\end{pr}


\subsection{Rational Functions}
\begin{df}
A rational function is one of the form
\begin{align*} f(x) & = \frac{P\left( x\right)}{Q\left( x\right)}  \end{align*}
where $P$ and $Q$ are polynomials.
\end{df}

\begin{ex} \begin{align*} f(x) & = \frac 1 {1-x^2}, & \dom\left( f\right) = \R \backslash\{1,-1\} \end{align*}
\end{ex}

\subsection{Exponential Function}
\begin{df}
The exponential function $\exp$ can be defined by several ways:
\begin{enumerate}
\item
As a power of $e$:
\begin{align*}
 \exp\left( x\right) = e^x
 \end{align*}
 Obviously, for this definition the number $e$ must be defined.
 \item
As a power series:
\begin{align*}
 \exp\left( x\right) = \sum_{m=0}^\infty \frac{x^m}{m!}
 \end{align*}
\item
By a ordinary differential equation (ODE):
\begin{align*}
 \frac{d}{dx} \exp\left( x\right) & = \exp\left( x\right) \\
 \exp\left( 0\right) & = 1
\end{align*}
\item
 As inverse of the natural logarithm:
\begin{align*}
 \exp^{-1}\left( x\right) & = \log\left( x\right) \\
 \log\left( x\right) & = \int_1^x \frac{du}{u}
\end{align*}
\item
As a limit:
\begin{align*}
\exp(x) = \lim_{n \to \infty} \left(1+ \frac x n \right)^n
\end{align*}
\end{enumerate}
\end{df}
\begin{pr}
\begin{align*}
\exp\left( x+y\right) = \exp\left( x\right) \cdot \exp\left( y\right)
\end{align*}
\end{pr} 

\subsection{Trigonometrical Functions}
\begin{df}
Similar to the exponential function, the trigonometrical functions cos and sin have several potential definitions:
\begin{enumerate}
\item
The elementary geometric definition at a right-angled triangle with a hypotenuse of length 1. 
\item
Definition through Polar form -- considering a point $p$ on a unit circle centred at the origin .
\item
As a power series:
\begin{align*}
\cos x & = 1- \frac{x^2}{2!} + \frac{x^4}{4!} + \frac{x^6}{6!} +\dots \\
\sin x & = x- \frac{x^3}{3!} + \frac{x^5}{5!} + \frac{x^7}{7!} +\dots
\end{align*}
\item
Through a system of ODEs:
\begin{align*}
\frac{d}{dx} \sin x & = \cos x \\
\frac{d}{dx} \cos x & = -\sin x \\
\sin 0 = 0&,  \quad \cos 0 =1
\end{align*}
\item
With the help of complex numbers:
\begin{align*}
\cos x = \frac{e^{ix}+e^{-ix}}{2}, \quad \sin x = \frac{e^{ix}-e^{-ix}}{2i}
\end{align*}
\end{enumerate} 
\end{df}
\begin{pr} \mbox \\
\begin{itemize}
\item
The addition formula:
\begin{align*}
\sin \left( x+y \right) & = \sin x \cos y + \cos x \sin y
\end{align*}
\item
 Shifting:
\begin{align*}
\sin\left( x+\frac{\pi}{2}\right) & = \cos x \\
\cos\left( x+\frac \pi 2\right) & = -\sin x \\
\sin\left( x+ \pi \right) & = \sin\left( x+ \frac \pi 2\right)+\frac \pi 2\\ 
 & = \cos\left( x+\frac \pi 2 \right) \\
 \sin \left( x+2\pi\right) & = \sin x
\end{align*}
\end{itemize}

\end{pr}
\begin{rk}
Special values which should be memorized are
\begin{align*}
x=0,\frac \pi 6, \frac \pi 4, \frac \pi 3, \frac \pi 2
\end{align*}
\end{rk}
\begin{df}
If a function $f$ has property $f\left( x+a\right) = f(x), ~ \forall x \in\dom ( f)$ it is called periodic. The period of $f$ is the smallest possible $a$ for which $f\left( x+a\right) = f(x), ~ \forall x \in\dom ( f)$ .
\end{df}

\begin{df}
Other trigonometric functions can be written as a combination of sine and cosine:
\begin{align*}
\sec x = \frac 1 {\cos x} \\
\operatorname{cosec} x = \frac 1 {\sin x} \\
\tan x = \frac{\sin x}{\cos x} \\
\cot x = \frac{\cos x}{\sin x} \\
\end{align*}
\end{df}
\subsection{Odd and Even Functions}
\begin{df}
A function $f$ is even if
\begin{align*}
  \forall x \in\dom ( f): \quad f\left( -x\right) = f(x)
\end{align*}
A function $f$ is odd if
\begin{align*}
 \forall x \in\dom ( f): \quad f\left( -x\right) = -f(x)
\end{align*}
\end{df}
\begin{rk} 
These definitions assume that $\dom\left( f\right)$ is symmetric which means $x \in \dom \left( f\right) ~ \Rightarrow ~-x \in \dom\left( f\right)$ 
\end{rk}
\begin{ex} $\sin x$ is odd, $\cos x$ is even.
 \end{ex}
 \begin{pr}
 A function can be neither odd nor even.
 However, any function can be split into a sum of even and odd functions
 \begin{align*}
 f(x) = \frac 1 2 \left(  f(x) + f\left( -x\right)\right) + \frac 1 2 \left(  f(x)-f\left( -x\right)\right)
 \end{align*}
 The odd and even part of a function are unique.
 \end{pr}
 \begin{ex}
 \begin{align*}
 e^x & = \frac 1 2\left( e^x + e^{-x}\right) + \frac 1 2 \left( e^x - e^{-x}\right) 
 \end{align*}
 \end{ex}
 \subsection{Hyperbolic Functions}
 \begin{df}
 \begin{align*}
 \cosh x & = \frac 1 2 \left(e^x +e^{-x} \right) \\
 \sinh x & = \frac 1 2 \left( e^x -e^{-x} \right)
 \end{align*}
 \end{df}
 \begin{pr} \mbox \\
 \begin{itemize}
 \item Addition theorem and derivatives:
 \begin{align*}
 \sinh (x+y)&  = \sinh x \cosh y + \cosh x \sinh y \\
 \sinh (x+y)&  = \sinh x \cosh y + \cosh x \sinh y \\
 \frac d {dx} \sinh x & = \cosh x \\
 \frac d {dx} \cosh x & = \sinh x
 \end{align*}
 
 \item
The hyperbolic functions can also be expressed through power series:
 \begin{align*}
 \cosh x & = 1 + \frac{x^2}{2!} + \frac{x^4}{4!} \dots \\
 \sinh x & = x + \frac{x^3}{3!} + \frac{x^5}{5!} + \dots
 \end{align*}
\item
 Similarly to the trigonometrical Pythagoras the following equation holds: 
 \begin{align*}
 \cosh^2 x - \sinh^2 x =1
 \end{align*}
 \end{itemize}
 \end{pr}

 
\begin{rk}
Origin of the name:
 \begin{align*}
 x & = \cosh t, & t \in \R \\
 y & = \sinh t \\
 x^2 -y^2 & = 1 
 \end{align*}
parametrizes a hyperbola. 
\end{rk}




\subsection{Inverse Functions}

\begin{df}
The inverse function $f^{-1}$, if it exists, is a function $f^{-1} : B \to A$ with the properties
\begin{align*}
f \left(f^{-1}(y) \right) & = y, & \forall y \in B \\
f^{-1}(f(x)) & = x, & \forall x \in A \\
\end{align*}
\end{df}

\begin{ex}
\begin{align*}
f(x) & = x^2 & A= [0,\infty) = B \\
f^{-1}(y) & = \sqrt y 
\end{align*}
\end{ex}
\begin{rk} \mbox \\
\begin{itemize}
\item
A necessary condition for a function to be invertible is that $f$ is injective (one-to-one). 
\begin{align*}
f(x_1) = f(x_2) \quad \Rightarrow \quad x_1 = x_2 
\end{align*}
or
\begin{align*}
f(x_1) \neq f(x_2) \quad \Leftarrow \quad x_1 \neq x_2
\end{align*}
Graphical test: $f$ is injective if its graph intersects any horizontal line at most once.

\item
The graph of the inverse $f^{-1}$ is the set of the points of the graph of $f$ with the $x$ and $y$ coordinates exchanged. The graph of $f^{-1}$ can be obtained by reflecting the graph of $f$ about the line $y=x$.

\item
If $f$ is strictly increasing (decreasing), it is injective.
\end{itemize}
\end{rk}

\begin{df}
$f$ is strictly increasing if
\begin{align*}
x_1 > x_2 \quad \Rightarrow \quad f(x_1) > f(x_2)
\end{align*}
$f$ is strictly decreasing if
\begin{align*}
x_1 > x_2 \quad \Rightarrow \quad f(x_1) < f(x_2)
\end{align*}
\end{df}
\begin{ex}
 The exponential function is strictly increasing. (proof in problem sheet) 
\end{ex}
\begin{rk}\mbox \\
 \begin{itemize} 
 \item
 Any even function $f$ is not injective if $\dom f \nsubseteq \{0\}$.
 \item
 Any periodic function is not injective either.
 \item
Therefore, the trigonometric functions
\begin{align*}
\sin, \cos, \tan
\end{align*}
are not invertible.

In order to inverse the $\sin$ function, restrict the domain to $\left[-\frac \pi 2, \frac \pi 2\right]$.

In order to inverse the $\cos$ funtion, restrict the domain to $[0, \pi]$.
\end{itemize}
\end{rk}
The inverse of the exponential function is called logarithm, $\log x$.

Anaytic treatment is sometimes possible. Require the existence of $f^{-1}(x)$ and 'Solve' $y=f(x)$ to obtain $x$ in terms of $y$, $x=f^{-1}(y)$.

\begin{ex} \mbox \\
\begin{itemize}
\item 
\begin{align*}
f(x) & = e^{- \frac 1 x} \\
x & = - \frac 1 {\log y}
\end{align*}
\item
inverse hyperbolic functions
\begin{align*}
f(x) & = \cosh x \\
f(x) & = \frac 1 2 \left( e^x + e^{-x} \right) \\
e^{2x} - 2ye^x +1 & = 0 \\
(e^x)^2 -2ye^x +1 & = 0 \\
e^x & = \frac{2y \pm \sqrt{4y^2-4}} 2 \\
e^x & = y \pm \sqrt{y^2-1} \\
x & = \log \left( y \pm \sqrt{y^2-1} \right)
\end{align*}
Restrict domain of $\cosh x$ to non-negative $x$.
\begin{align*}
x & = \log ( y + \sqrt{y^2-1} \\
\cosh^{-1} x & = \log \left( x + \sqrt{x^2-1} \right)
\end{align*}
\begin{align*}
\sinh^{-1} x = \log \left( x + \sqrt{1+x^2} \right)
\end{align*}
\end{itemize}
\end{ex}


\subsubsection{Derivatives of Inverse Functions}
The slope of the inverse function is $\frac 1 {f'(a)}$, the reciprocal of slope of the original function.
\begin{ex}
$f(x) = e^x = y$ so that $x = \log y$.
\begin{align*}
\frac{dx}{dy} = \left( \frac{dy}{dx} \right)^{-1} = \frac{1}{e^x} = \frac 1 y
\end{align*}
or
\begin{align*}
\frac{d}{dy} \log y = \frac 1 y
\end{align*}
\end{ex}


\subsubsection{Inverse Trigonometrical functions}
We are going to differentiate $\sin^{-1}$, $\tan^{-1}$. We set $y= \sin x $ so that $x = \sin^{-1} y$.
\begin{align*}
\frac{dx}{dy} = \left( \frac {dy}{dx} \right)^{-1}  = \frac 1 {\cos x} & = \frac 1 {\sqrt{1-\sin^2 x }} \\
& = \frac 1 {\sqrt{1-y^2}} \\
\frac d {dx} \sin^{-1} x & = \frac 1 {\sqrt{1 -x^2}}
\end{align*}
Similarly,
\begin{align*}
\frac{d}{dx} \tan^{-1} x = \frac {1}{1+x^2}
\end{align*}
 
\begin{pr}
 \begin{align*}
 \sin^{-1} x + \cos^{-1} x = \frac \pi 2 
 \end{align*}
\end{pr}
\begin{ex} Find the mistake in the following proof.
 \begin{align*}
 \sin \left(x + \frac \pi 2 \right) & = \cos x \\
 x & = \cos ^{-1} y \\
 \sin \left( \cos^{-1} y + \frac \pi 2 \right) & = \cos \left( \cos^{-1} y \right) \\
 \sin \left( \cos^{-1} y + \frac \pi 2 \right) & = y \\
 \cos^{-1} y + \frac \pi 2 & = \sin^{-1} y \\
 \sin^{-1}-\cos^{-1} y & = \frac \pi 2
 \end{align*}
\end{ex} 
 
 
 
 
 
 
 
 
 
\section{Graphs}
\begin{df}
The graph of a function $f$ is defined by $y=f(x)$
\end{df}
\begin{df}
$a \in \dom (f)$ is a \emph{stationary point} if $f'(a) = 0$
\end{df}
\begin{rk}
A stationary point can be a local minimum, a local maximum or a point of inflection with horizontal tangent.
\end{rk}
Suppose $a$ is a stationary point.
\begin{enumerate}
\item
If $f''(a) > 0$, then $a$ is a local minimum
\item
If $f''(a) < 0$, then $a$ is a local maximum
\item
If $f''(a) = 0$ gives no information.
\end{enumerate}
This test is called the 2nd Derivative Test.
\begin{ex}
\begin{align*}
f(x) & = x^4 \\
f'(x) & = 4x^3
\end{align*}
$x=0$ is a stationary point because
$f''(x) = 12 x^2 =0$.
\end{ex}

\emph{Geometrical definition.}
A point of inflection is a point where the graph crosses its own tangent

A sufficient condition for a point of inflection $(p_0, I)$ is:
If $f''(a) = 0 $ and $f'''(0) \neq 0$, then $a$ is a point of inflection.
This is not a necessary condition.

\begin{ex}
\begin{align*}
f(x) & = x^5 \\
f'(x) & = 5x^4 \\
f''(x) & = 20x^3 \\
f'''(x) & = 60x^2
\end{align*}
The sufficient condition does not work for this example at $x=0$ but $(0,0)$ is a point of inflection.
\end{ex}
A point of inflection is not necessarily a stationary point.
\begin{ex}
\begin{align*}
f(x) & = x^4 -2x^2 \\
f'(x) = 4x^3 -4x & = 4x \left(x^2-1 \right) \\
& = 4x(x-1)(x+1)
\end{align*}
There are 3 stationary points at $x=0$ and $x = \pm 1$.
\begin{align*}
f''(x) & = 12 x^2 -4 \\
f''(0) & = -4 <0
\end{align*}
So $x=0$ is a local maximum. Furthermore,
\begin{align*}
f''(\pm 1) = 12 - 4 = 8 > 0
\end{align*}
Hence, $x = \pm 1$ is a local minimum. Consider the following equation to find points of inflection
\begin{align*}
f''(x) = 12x^2 - 4 = 0 
\end{align*}
This holds if $x^2 = \frac 1 3$. i.e. $x = \pm 1 /\sqrt 7$ are points of inflection since $f'''(x) = 24x \neq 0$ at these points.
\end{ex}

\subsection{Curve sketching}
There is no correct way to sketch the graph of a function -- in some cases the graph is too complicated to sketch it by hand. In this case try using a computer.
(e.g. Riemann's Lacunary Fourier series.)
However, the following often helps:
\begin{enumerate}
\item
Does the graph have any special features (e.g. odd, even or periodic)?
\item
Does the graph intersect the $x$ or $y$ axes?
\item
Does the graph have stationary points or points of inflection?
\item
Does the graph have linear asymptotes?
\end{enumerate}
Rational functions often have linear asymptotes.
\begin{ex}
\begin{align*}
f(x) = \frac{x^3}{1-x^2}
\end{align*}
\begin{itemize}
\item
At $x = \pm 1$ the graph has vertical asymptotes. 
\item
For $x \to \pm \infty$ the graph has the linear asymptote $y= x $.
\item
For $x$ small $f(x) \approx x^3$.
\end{itemize}
\end{ex}


\subsubsection{Polar Coordinates}
We can represent curves via functions, equations or parametrically -- yet another way is through polar coordinates.
The Idea is to replace the cartesian coordinates $x$ and $y$ with polar coordinates $r$ and $\theta$. $r$ is the distance to the origin, $\theta$ is the angle measured anti-clockwise from the x axis. Replacing $\theta$ by $\theta +2 \pi$ has no effect.
\begin{align*}
x = r \cos \theta, \qquad y = r \sin \theta
\end{align*}

Now we are able to represent curves using equations involving $r$ and $\theta$ instead of $x$ and $y$.

\begin{ex}
\begin{itemize}
\item
If $r$ is a constant greater than 0, the equations represent a circle.
\item
Let $l$ be a positive constant and $e$ be a non-negative constant.
\begin{align*}
r = \frac l {1+e \cos \theta}
\end{align*}
gives a conic section where $e$ is the 'eccentricity'.
\end{itemize}
\end{ex}

\subsubsection{Conic sections}

In general consider $A x^2 + Bxy +Cy^2 + Dx + Ey + F = 0$, where $A, B, C, D, E, F$ are constants.

Degenerate cases are:
\begin{itemize}
\item
point $x^2 + y^2 = 0$ 
\item
line $y=0$
\item
two lines $x^2 - y^2 = 0 \Leftrightarrow x = \pm y$
\item
two parallel lines
\end{itemize}

All other possibilities are of three types
ellipse, parabola, hyperbola.
\begin{df}
An \emph{ellipse} is a curve defined by 
\begin{align*}
\frac {x^2}{a^2} + \frac{y^2}{b^2} = 1
\end{align*}
or any translation or rotation of this curve. $a=b$ reduces to a circle.
\end{df}


\begin{df}
A Parabola is a curve of the form
\begin{align*}
y = a x^2
\end{align*}
or any translation or rotation of this curve.
\end{df}

\begin{df}
A Hyperbola is a curve of the form
\begin{align*}
\frac {x^2}{a^2} - \frac{y^2}{b^2} = 1
\end{align*}
or any translation or rotation of this curve.
\end{df}

The equation
\begin{align*}
r = \frac l {1+e \cos \theta}
\end{align*}
\begin{itemize}
\item
gives us an ellipse for $0 \leq e < 1$.
\item
gives us a parabola for $e=1$.
\item
gives us a hyperbola for $e>1$.
\end{itemize}
Set $l=1$ for all cases:
To obtain the equation for a parabola set $e =1$:
\begin{align*}
r = \frac 1 {1+ \cos \theta}
\end{align*}
To obtain the equation for a parabola set $e = \frac 1 2$:
\begin{align*}
r = \frac 1 {1+ \frac 1 2 \cos \theta}
\end{align*}
To obtain the equation for a parabola set $e =2$:
\begin{align*}
r & = \frac 1 {1+ 2 \cos \theta} \\
-\frac{2 \pi} 3 & < \theta < \frac{2 \pi} 3
\end{align*}

There are two different conventions for dealing with negative $r$:
\begin{enumerate}
\item
Discard any $\theta$ values leading to negative $r$.
\item
Retain $\theta$ values leading to negative $r$.
\begin{align*}
x = r \cos \theta \qquad y = r \sin \theta
\end{align*}
Allow $r$ to be negative. In case of $r$ negative, flip the sign of $r$, i.e. flip the sign of $x$ and $y$. This is equivalent to shifting $\theta$ by $\pi$ (or $-\pi$)
\begin{align*}
\sin(\theta \pm \pi) & = \sin \theta \\
\cos(\theta \pm \pi) & = \cos \theta
\end{align*}
Using this prescription
\begin{align*}
r = \frac l {1+ e \cos \theta}
\end{align*}
gives a full hyperbola (both branches) for $e>1$.
\end{enumerate}



